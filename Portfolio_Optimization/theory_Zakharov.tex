\documentclass[12pt]{article}
 \usepackage[margin=1in]{geometry}
\usepackage{amsmath, amsthm, amssymb, amsfonts, enumitem, fancyhdr, color, comment, graphicx, environ, fancyhdr}
\pagestyle{fancy}
\usepackage{cmap}
\usepackage[T2A]{fontenc}
\usepackage[utf8]{inputenc}
\usepackage[english, russian]{babel}
\usepackage{caption}
\usepackage{indentfirst}
% \usepackage{hyperref}

% \usepackage{color} \definecolor{darkgreen}{rgb}{0,0,0}
% \usepackage[unicode,colorlinks,filecolor=blue,citecolor=darkgreen,pagebackref]{hyperref}

\newtheorem*{exersize}{Упражнение}
\newtheorem*{lemma}{Lemma}
\newtheorem{theorem}{Теорема}[section]
\newtheorem*{consequence}{Следствие}
\newtheorem*{statement}{Statement}
\newtheorem{property}{Свойство}
\newtheorem*{fact}{Fact}

\theoremstyle{definition}
\newtheorem{definition}{Определение}
\newtheorem*{problem}{Problem}
\newtheorem*{example}{Example}

\theoremstyle{remark}
\newtheorem*{remark}{Remark}
\setlength{\parindent}{5ex}
\setlength{\headheight}{35pt}
\lhead{Kirill Zakharov}

\begin{document}
\section*{Securities portfolio modeling}
\subsection*{1. Классический портфель}
$x = (x_1, ..., x_n); r=(r_1,...,r_n); l=(1,...,1)\in \mathbb{R}^n$\\
Рассмотрим следующую квадратичную функцию
\begin{equation}
  f(X)=\sum_{i=1}^nr_ix_i + \frac{1}{2}\sum_{i=1}^n\sum_{j=1}^n \gamma_{ij}x_ix_j
\end{equation}
\begin{equation}
  f(X)=r x^T+\frac{1}{2}x\Sigma x^T
\end{equation}

$\Sigma$ - положительно полуопределенная матрица.
\begin{definition}
  Портфель называется эффективным по дисперсии, если для фиксированного дохода портфеля $r_p$, не существует портфеля с меньшей дисперсией.
\end{definition}
\begin{definition}
  Портфель называется эффективным по доходности, если для фиксированной дисперсии портфеля $\sigma_p^2$, не существует портфеля с большим доходом.
\end{definition}
Для получения максимального дохода, будем решать следующую оптимизационную задачу
\begin{equation}
  \max\limits_{X}\big\{ r\cdot x^T \;\;\big| l \cdot x^T=1; x\Sigma x^T \leq \tilde{\sigma}_p; x_i \geq 0\big\}
\end{equation}

А для минимизации риска
\begin{equation}
  \min\limits_{X}\big\{ x\Sigma x^T \;\;\big| l \cdot x^T=1;r\cdot x^T  \geq \tilde{r}_p; x_i \geq 0\big\}
\end{equation}
\subsection*{2. Обобщенная функция. Граница эффективности}
Пусть $t$ - неотрицательный параметр. Будем решать следующую задачу
\begin{equation}
  \min\limits_{X}\big\{-t\cdot r\cdot x^T +\frac{1}{2} x\Sigma x^T \;\;\big| l \cdot x^T=1; x_i \geq 0\big\}
\end{equation}

\begin{definition}
  Портфель называется параметрически-эффективным, если он является оптимальным решением задачи (5) для некоторого неотрицательного параметра $t$.
\end{definition}

\begin{definition}
  Оптимальными условиями для квадратичной функции являются $Ax_0=b$ и существование вектора $u$, такого что $-\nabla f(x_0)=u A^T$.
\end{definition}
Тогда получаем $t\cdot r - \Sigma x^T = u \cdot l^T \land l\cdot x^T=1$. Решаем по x.

\begin{align}
&x^T=-u\Sigma^{-1}l^T + t\Sigma^{-1}r^T\\
&l\cdot x^T = -u\cdot l\Sigma^{-1}l^T + t\cdot l\Sigma^{-1}r^T = 1\\
&u = \frac{-1}{l\Sigma^{-1}l^T} + t\frac{l\Sigma^{-1}r^T}{l\Sigma^{-1}l^T}
\end{align}
Подставим найденный вектор $u$ в уравнение (6).
\begin{align}
  &x^T\equiv x^T(t)=\frac{\Sigma^{-1}l^T}{l\Sigma^{-1}l^T} - \Sigma^{-1}l^T\cdot t\frac{l\Sigma^{-1}r^T}{l\Sigma^{-1}l^T} + t\cdot \Sigma^{-1}r^T=\\
  &=\frac{\Sigma^{-1}l^T}{l\Sigma^{-1}l^T} + t\big(\Sigma^{-1}r^T - \Sigma^{-1}l^T\cdot t\frac{l\Sigma^{-1}r^T}{l\Sigma^{-1}l^T}\big)\notag
\end{align}
Пусть $h_0=\dfrac{\Sigma^{-1}l^T}{l\Sigma^{-1}l^T},\;h_1=\Sigma^{-1}r^T - \Sigma^{-1}l^T\cdot t\dfrac{l\Sigma^{-1}r^T}{l\Sigma^{-1}l^T}$. Тогда
получим следующую зависимость от параметра $t$.
\begin{equation}
  x^T(t)=h_0+th_1
\end{equation}
Теперь определим доходность портфеля и риск.
\begin{align}
  &r_p=r\cdot x^T(t)=rh_0+rth_1\\
  &\sigma_p^2=(h_0+th_1)^T\Sigma (h_0+th_1)=h_0^T\Sigma h_0 + th_1^T\Sigma h_0 + th_0^T\Sigma h_1 + t^2h_1^T\Sigma h_1=\\
  &=h_0^T\Sigma h_0 + 2th_1^T\Sigma h_0 + t^2h_1^T\Sigma h_1 \notag
\end{align}
Введем дополнительные обозначения.
\begin{align}
  &\alpha_0=r\cdot h_0\notag\\
  &\alpha_1=r \cdot h_1\notag\\
  &\beta_0 = h_0^T\Sigma h_0\\
  &\beta_1 = h_1^T\Sigma h_0\notag\\
  &\beta_2 = h_1^T\Sigma h_1\notag
\end{align}
Тогда получим $r_p=\alpha_0+t\alpha_1,\; \sigma_p^2=\beta_0+2t\beta_1+t^2\beta_2$. Можно показать, что $\beta_1=0, \alpha_1=\beta_2$. Выразим $t$.
\begin{align}
  &r_p\to:\;t=\frac{r_p-\alpha_0}{\alpha_1}\\
  &\sigma_p^2\to:\;t^2=\frac{\sigma_p^2-\beta_0}{\beta_2}=\frac{\sigma_p^2-\beta_0}{\alpha_1}
\end{align}
Возведем в квадрат (14) и приравняем к (15). Получим следующее соотношение, называемое границей эффективности (efficient frontier).
\begin{equation}
  \frac{(r_p-\alpha_0)^2}{\alpha_1}=\sigma_p^2-\beta_0
\end{equation}

\subsection*{3. Коэффициент Шарпа. The capital market line}
Пусть $r$ - доходность по безрисковому активу. Тогда доходность портфеля определяется по формуле (17).
$\mu=\begin{pmatrix} \mu_1\\ \vdots \\ \mu_n \end{pmatrix},\;x=\begin{pmatrix} x_1\\ \vdots \\ x_n \end{pmatrix}$
\begin{equation}
  r_p = \mu_1x_1+...+\mu_nx_n+r x_{n+1}=\begin{pmatrix} \mu^T&r \end{pmatrix}\cdot \begin{pmatrix} x\\ x_{n+1} \end{pmatrix}
\end{equation}
И риск портфеля
\begin{equation}
 COV=x^T\Sigma x=\begin{pmatrix} x\\ x_{n+1} \end{pmatrix}^T \begin{bmatrix} \Sigma&\mathbb{O}\\\mathbb{O}^T&0 \end{bmatrix}
   \begin{pmatrix} x\\ x_{n+1} \end{pmatrix}
\end{equation}
$COV\geq 0, \Sigma>0$\\
Будем решать следующую задачу
\begin{equation}
  \min\limits_{x}\bigg\{ -t\begin{pmatrix} \mu^T&r \end{pmatrix}\cdot \begin{pmatrix} x\\ x_{n+1}\end{pmatrix}
  +\frac{1}{2} \begin{pmatrix} x\\ x_{n+1} \end{pmatrix}^T \begin{bmatrix} \Sigma&\mathbb{O}\\\mathbb{O}^T&0 \end{bmatrix}
    \begin{pmatrix} x\\ x_{n+1} \end{pmatrix}\;\;\bigg| l\cdot x+x_{n+1}=1 \bigg\}
\end{equation}
\end{document}
